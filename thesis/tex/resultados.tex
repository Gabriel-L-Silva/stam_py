\label{ch:resultados}

Neste capítulo, apresentamos os resultados obtidos com o método proposto. Embora a implementação tenha sido inicialmente concebida para ser interativa, a complexidade do tratamento de malhas arbitrárias nos levou a adotar uma abordagem não-interativa. Os resultados são avaliados tanto quantitativamente, através da análise do divergente do campo de velocidades, quanto qualitativamente, por meio da visualização da simulação.

\section{Análise do Divergente}
\label{sec:divergente}

Uma propriedade fundamental em simulações de fluidos incompressíveis é a condição de divergente nulo no campo de velocidades. Nos experimentos subsequentes, se apresenta a distribuição espacial do divergente na malha em um instante específico da simulação em forma de um gráfico tridimensional, assim como se mostra a evolução temporal do divergente máximo ao longo da simulação em forma de um gráfico bidimensional.

A análise destes resultados revela um aumento gradual do divergente ao longo do tempo, particularmente pronunciado próximo às fronteiras do domínio. Este comportamento está diretamente relacionado às limitações do nosso método em regiões de geometria complexa.

\section{Visualização da Simulação}
\label{sec:visualizacao}

Para avaliar qualitativamente o método proposto, realizamos três experimentos distintos de simulação de fumaça em diferentes geometrias. As Figuras~\ref{fig:exp1}, \ref{fig:exp2} e \ref{fig:exp3} apresentam a evolução temporal de cada experimento em três instantes: inicial, intermediário e final (de cima para baixo). Em todo começo de simulação é esperado que tenhamos o divergente alto, pois estamos injetando fumaça no campo. Esse injetor é desligado na metade da simulação, por isso a queda repentina no gráfico de divergente máximo pelo tempo.

\begin{figure}[p]
   \centering
   \includegraphics[width=0.85\textwidth]{images/ghost_comp_frame_0.png}
   \includegraphics[width=0.85\textwidth]{images/ghost_comp_frame_1.png}
   \includegraphics[width=0.85\textwidth]{images/ghost_comp_frame_2.png}
   \caption{Primeiro experimento: Comparação da eficácia da técnica de nós fantasma.}
   \label{fig:exp1}
\end{figure}

\begin{figure}[p]
   \centering
   \includegraphics[width=0.85\textwidth]{images/tunisia_frame_0.png}
   \includegraphics[width=0.85\textwidth]{images/tunisia_frame_1.png}
   \includegraphics[width=0.85\textwidth]{images/tunisia_frame_2.png}
   \caption{Segundo experimento: simulação da fumaça em uma geometria simples irregular.}
   \label{fig:exp2}
\end{figure}

\begin{figure}[p]
   \centering
   \includegraphics[width=0.85\textwidth]{images/djobuti_frame_0.png}
   \includegraphics[width=0.85\textwidth]{images/djobuti_frame_1.png}
   \includegraphics[width=0.85\textwidth]{images/djobuti_frame_2.png}
   \caption{Terceiro experimento: simulação em um domínio com fronteiras irregulares complexas.}
   \label{fig:exp3}
\end{figure}

No primeiro experimento (Figura~\ref{fig:exp1}), comparamos o impacto da solução de nós fantasma simplificada que adotamos. Especialmente nesse exemplo, não utilizamos o algoritmo de interseção raio-polígono. Como podemos analisar pelos gráficos, temos uma estabilidade muito maior no valor do divergente a cada passo de tempo e também analisamos que os pontos onde o divergente diferente de zero passaram de estar por toda a malha para estarem só na fronteira. 

O segundo experimento (Figura~\ref{fig:exp2}) apresenta uma geometria mais complexa, testando a capacidade do método em lidar com fronteiras irregulares. Nota-se que, mesmo com a maior complexidade geométrica, o método consegue capturar os principais fenômenos físicos esperados. Aqui, começam a se manifestar algumas das limitações do método, como os efeitos da perda de massa e instabilidades próximas às fronteiras mais complexas.

No terceiro experimento (Figura~\ref{fig:exp3}), exploramos uma configuração ainda mais desafiadora, com fronteiras altamente irregulares. Nesse experimento, podemos analisar que o divergente não se estabiliza em um valor baixo, acarretando na perda de massa ainda mais rápida. Mas ainda assim temos uma qualidade no escoamento do fluido.

Em todos os casos, é possível observar a capacidade do método em reproduzir comportamentos físicos qualitativamente corretos, como:

\begin{itemize}
   \item Formação e evolução de estruturas vorticais
   \item Interação adequada com as fronteiras do domínio
   \item Difusão realista da densidade da fumaça
\end{itemize}

No entanto, também são evidentes alguns artefatos numéricos, particularmente em regiões de alta curvatura da fronteira e nos estágios mais avançados da simulação, onde a perda de massa se torna mais pronunciada.

\section{Conservação de Massa}
\label{sec:conservacao}

Um desafio significativo identificado durante os testes foi a perda gradual de massa ao longo da simulação, como pode ser observado em todos os experimentos. Este fenômeno é principalmente atribuído a dois fatores:

\begin{enumerate}
   \item Imprecisões na aproximação das condições de contorno junto à fronteira;
   \item Instabilidades numéricas em regiões onde nós da malha estão muito próximos.
\end{enumerate}

A escolha da função de base radial poliarmônica mostrou-se particularmente sensível a estas configurações geométricas, contribuindo para a instabilidade do método em fronteiras complexas.