\label{secao:revisao}
Este capítulo apresenta uma revisão bibliográfica focada em métodos de simulação de fluidos aplicados à computação gráfica, com ênfase em animação de fumaça. As primeiras abordagens para animação de fumaça foram baseadas em métodos não-físicos, sem a utilização das ENS, por meio da animação de texturas \cite{Huang2015}. Dos métodos baseados em física, o trabalho de Jos Stam~\cite{Stam1999} tem grande relevância, pois introduziu na literatura o método de advecção semi-lagrangiana, que aumentou o desempenho na simulação de fluidos, dado que é uma solução simples e incondicionalmente estável para o problema da advecção. No entanto, o método apresenta dissipação numérica significativa. Além disso, o método proposto por Stam faz uso de grades co-localizadas. A advecção semi-lagrangiana utiliza os valores de velocidades no centro da célula para retroceder no passo de tempo e copiar a velocidade da posição do passo anterior para a célula atual.

Em 2003, Jos Stam publicou um artigo sobre simulação de fluidos para jogos, focando no desempenho e detalhes de implementação \cite{Stam2003}. Neste trabalho, o autor apresenta mais detalhes sobre a implementação da malha, explicando o método de células fantasmas para auxiliar na computação de operadores diferenciais por meio de FD. O trabalho aborda as condições de fronteira de Neumann, em que a componente normal da velocidade das células fantasmas é uma cópia das velocidades dentro da grade, porém com o sinal trocado, fazendo com que a componente perpendicular da velocidade do campo na fronteira seja nula, enquanto a componente tangencial permanece livre. Também é apresentada a condição de fronteira de Dirichlet para o campo de pressão, onde os valores são copiados mantendo o sinal. Como nosso foco não é em grades regulares, para leitores interessados em se aprofundar nesse tópico recomendamos o \textit{survey}~\cite{Huang2015}.

Os trabalhos de Stam mencionados utilizam domínios de simulação simples: um deles utiliza fronteira periódica em uma \sigla{AABB}{\emph{Axis Aligned Bounding Box}} \cite{Stam1999}, o outro somente uma AABB \cite{Stam2003}. Este tipo de fronteira facilita a aplicação das condições de contorno mencionadas anteriormente. Porém, malhas regulares têm dificuldade para representar domínios com fronteiras complexas. Logo, a melhor escolha para trabalhar com tais fronteiras é o uso de malhas triangulares (2D) e tetraedrais (3D) \cite{Feldman2005}. A abordagem deste artigo é utilizar uma malha híbrida, em que é possível aproveitar os benefícios dos dois tipos de malhas. A parte regular é usada para preencher espaços vazios (sem fluido), enquanto a parte triangular é usada nas fronteiras, preservando mais detalhes durante a simulação. Nas células triangulares é utilizada uma representação de malha deslocada, com a restrição de cada triângulo conter o próprio circuncentro.

Como malhas não-estruturadas (triangulares em 2D e tetraedrais em 3D) se adaptam melhor às fronteiras complexas, o uso de malhas dinâmicas torna-se fundamental quando há obstáculos que podem se mover durante a simulação. Essa é a abordagem usada em \cite{Klingner2006}, que utiliza malhas tetraedrais dinâmicas. Para tais malhas, o método numérico utilizado é o de FV. A malha é recalculada a cada passo de tempo da simulação, exigindo que sua geração seja eficiente e rápida. O tipo de geração de malha escolhido foi o de Delaunay, cujo refinamento é complexo, pois deve manter a conformidade da malha. Foi necessário adaptar a advecção semi-lagrangiana para acomodar a troca de malhas durante o passo de tempo. Além disso, a reconstrução da malha utiliza uma técnica na qual a malha se adapta bem a obstáculos e é refinada com base em critérios arbitrários.


Métodos com malha baseada em árvores são amplamente utilizados em abordagens adaptativas \cite{quadtree1, quadtree2, Nakanishi2020}, pois esta estrutura facilita o aumento ou a diminuição da resolução da malha em regiões específicas (refinamento). A principal desvantagem é que o refinamento de \textit{quadtrees} (2D) e \textit{octrees} (3D) pode gerar malhas não conformes. Isto causa complicações nos métodos numéricos mais tradicionais (FD e FV), criando a necessidade de determinar independentemente os vizinhos nestes pontos de não conformidade. Uma solução é usar RBF-FD para realizar a interpolação livre de malha. Embora existam diversos trabalhos que utilizam RBF-FD para resolução de \sigla{EDP}{Equações Diferenciais Parciais} \cite{RBF1,RBF2}, incluindo as ENS \cite{Mahmood2019}, o primeiro artigo a usar este método para simular fluidos na área de computação gráfica \cite{Nakanishi2020} tem como principais contribuições: uma nova técnica para simulação de fluido completamente adaptativa com abordagem híbrida (euleriana e lagrangiana); e o uso de RBF-FD para aproximar os operadores diferenciais em árvores arbitrárias, enquanto a adaptatividade é calculada durante a simulação com o intuito de ajustar a grade à interface do líquido.

% Please add the following required packages to your document preamble:
%\usepackage{booktabs}   
\begin{sidewaystable}
\caption{Resumo dos artigos revisados.}
\begin{tabular}{@{}cccccccc@{}}
\toprule
Artigo                                                               & Malha                              & Célula          & Método Numérico & Adaptativa & Dinâmica & Conforme & Dim.\\ \midrule
\hline\\[.125cm]
\cite{Stam1999} \& \cite{Stam2003} & Co-localizada & Quad            & FD              & Não        & Não      & Sim      & 2D/3D \\[.5cm]

\cite{Feldman2005}                                  & deslocada  & Tri/Quad             & FV              & Não        & Sim      & Sim      & 2D/3D \\[.5cm]

\cite{Klingner2006}                                 & deslocada  & Tet         & FV              & Não        & Sim      & Sim      & 3D \\[.5cm]

\cite{Nakanishi2020}                                & deslocada  & Quadtree/Octree & RBF-FD          & Sim        & Não      & Não      & 2D/3D \\ [.5cm]

Nossa proposta                              & deslocada  & Tri & RBF-FD          & Sim        & Não      & Não      & 2D \\ [.5cm]
\bottomrule
\end{tabular}
\subcaption*{Fonte: Elaborada pelo autor.}
\label{tab:revisao}
\end{sidewaystable}

Durante a pesquisa do acervo para construir esta revisão bibliográfica, não foi encontrado na literatura um método capaz de realizar animação de fumaça em uma malha não-estruturada e não-conforme. Apesar de encontrarmos resultados relevantes \cite{RBF1,RBF2}, nenhum deles satisfaz todos os requisitos apresentados na Tabela~\ref{tab:revisao}. O trabalho que mais se aproxima de todos os requisitos é o método proposto por Nakanishi et al.~\cite{Nakanishi2020}. Entretanto, o foco deste trabalho é construir uma grade adaptativa, e a estrutura de dados é baseada em árvore. Nossa proposta é preencher essa lacuna na literatura e desenvolver uma simulação de fluidos em animação computacional com uma malha triangular bidimensional não-conforme.