\label{secao:revisao}
Este capítulo apresenta uma revisão bibliográfica focada em métodos de simulação de fluidos aplicados à computação gráfica, com ênfase em animação de fumaça. Inicialmente, a animação de fumaça foi baseada em métodos não-físicos, sem a utilização das ENS. Isso foi possível por meio da animação de texturas \cite{Huang2015}. Dos métodos baseados em física, o trabalho de Jos Stam~\cite{Stam1999} tem grande relevância, pois introduziu na literatura o método de advecção semi-lagrangiana, que aumentou o desempenho na simulação de fluidos, dado que é uma solução simples e incondicionalmente estável para o problema da advecção. Porém, o método sofre gravemente em relação à dissipação numérica. Além disso, o método proposto por Stam faz o uso de grades co-localizadas. A advecção semi-lagrangiana usa os valores de velocidades no centro da célula para retroceder no passo de tempo e copiar a velocidade da posição do passo anterior para a célula atual.

Em 2003, Jos Stam publicou um artigo sobre simulação de fluidos para jogos, focando no desempenho e detalhes de implementação \cite{Stam2003}. Aqui ele mostra mais detalhes sobre a implementação da malha, explicando o método de células fantasmas para auxiliar na computação de operadores diferenciais por meio de FD. Além de trabalhar com as condições de fronteira de Neumann, em que a componente normal da velocidade das células fantasmas são cópias das velocidades dentro da grade, porém com o sinal trocado, fazendo com que a componente perpendicular da velocidade do campo na fronteira seja nula, enquanto a componente tangencial permanece livre. E a condição de fronteira de Dirichlet para o campo de pressão, onde os valores são copiados, mantendo o sinal. Como o nosso foco não é em grades regulares, caso queira se aprofundar mais nesse tópico recomendamos o \textit{survey}~\cite{Huang2015}.

Os trabalhos de Stam mencionados utilizam domínios de simulação simples: um deles utiliza fronteira periódica em uma \sigla{AABB}{\emph{Axis Aligned Bounding Box}} \cite{Stam1999}, o outro somente uma AABB \cite{Stam2003}. Este tipo de fronteira facilita a aplicação das condições de contorno mencionadas anteriormente. Porém, malhas regulares têm dificuldade para comportar domínios com fronteiras complexas. Logo, a melhor escolha para trabalhar com estas fronteiras é o uso de malhas triangulares (2D) e tetraedrais (3D) \cite{Feldman2005}. A abordagem deste artigo é utilizar uma malha híbrida, em que é possível aproveitar dos benefícios dos dois tipos de malhas. A parte regular é usada para preencher espaços vazios (sem fluido), já a parte triangular é usada nas fronteiras, perdendo menos detalhes durante a simulação. Nas células triangulares é utilizada uma representação de malha deslocada, com a restrição de cada triângulo conter o próprio circuncentro.

Como malhas não-estruturadas (triangulares em 2D e tetraedrais em 3D) se adaptam melhor às fronteiras complexas, o uso de malhas dinâmicas torna-se fundamental quando temos obstáculos que podem se mover durante a simulação. Essa é a abordagem usada em \cite{Klingner2006}, que utiliza malhas tetraedrais dinâmicas. Como a malha não possui uma estrutura regular, o método numérico utilizado é o de FV. A malha é recalculada a cada passo de tempo da simulação, exigindo que a geração da malha seja eficiente e rápida. O tipo de geração de malha escolhido foi o de Delaunay, cujo refinamento é complexo, pois deve manter a conformidade da malha. Foi necessário alterar a advecção semi-lagrangiana para se adaptar à troca de malhas durante o passo de tempo. Além disso, a reconstrução da malha utiliza uma técnica na qual a malha se adapta bem a obstáculos e é refinada baseada em critérios arbitrários.


Métodos com malha baseada em árvores são famosos em abordagens adaptativas \cite{quadtree1, quadtree2, Nakanishi2020}, pois esta estrutura tem facilidade em aumentar ou diminuir a resolução da malha em regiões especificas (refinamento). A principal desvantagem é que o refinamento de \textit{quadtrees} (2D) e \textit{octrees} (3D) podem gerar malhas não conformes. Isto causa complicações nos métodos numéricos mais tradicionais (FD e FV), criando a necessidade de resolver independentemente a determinação dos vizinhos nestes pontos de não conformidade. Uma solução é usar RBF-FD para conseguir fazer a interpolação livre de malha. Apesar de existirem diversos trabalhos que usam RBF-FD para resolução de uma \sigla{EDP}{Equações Diferenciais Parciais} \cite{RBF1,RBF2}, incluindo as ENS \cite{Mahmood2019}. O primeiro artigo à usar este método para simular fluidos na área de computação gráfica \cite{Nakanishi2020} tem como principais contribuições: uma nova técnica para simulação de fluido completamente adaptativa com abordagem híbrida (euleriana e lagrangiana). Uso de RBF-FD para aproximar os operadores diferenciais em árvores arbitrárias, enquanto a adaptatividade é calculada durante a simulação com o intuito de ajustar a grade à interface do líquido.

% Please add the following required packages to your document preamble:
%\usepackage{booktabs}   
\begin{sidewaystable}
\caption{Resumo dos artigos revisados.}
\begin{tabular}{@{}ccccccc@{}}
\toprule
Artigo                                                               & Malha                              & Célula          & Método Numérico & Adaptativa & Dinâmica & Conforme\\ \midrule
\hline\\[.125cm]
\cite{Stam1999} \& \cite{Stam2003} & Co-localizada & Quad            & FD              & Não        & Não      & Sim      \\[.5cm]

\cite{Feldman2005}                                  & deslocada  & Tri             & FV              & Não        & Sim      & Sim      \\[.5cm]

\cite{Klingner2006}                                 & deslocada  & Híbrido         & FV              & Sim        & Não      & Sim      \\[.5cm]

\cite{Nakanishi2020}                                & deslocada  & Quadtree/Octree & RBF-FD          & Sim        & Não      & Não      \\ [.5cm]

Nossa proposta                              & deslocada  & Tri & RBF-FD          & Sim        & Não      & Não      \\ [.5cm]
\bottomrule
\end{tabular}
\subcaption*{Fonte: Feito pelo autor.}
\label{tab:revisao}
\end{sidewaystable}

Durante a pesquisa do acervo para construir esta revisão bibliográfica, não foi encontrado na literatura um método capaz de fazer animação de fumaça em uma malha não-estruturada e não-conforme. Apesar de encontrar resultados relevantes \cite{RBF1,RBF2}, nenhum deles satisfaz todos os requisitos apresentados na Tabela~\ref{tab:revisao}. O trabalho que mais cumpre todos os requisitos é o método proposto por Nakanishi et al.~\cite{Nakanishi2020}. Entretanto, o foco deste trabalho é fazer uma grade adaptativa, e a estrutura de dados é baseada em árvore. A nossa proposta é preencher essa lacuna na literatura e produzir uma simulação de fluidos em animação computacional com uma malha de triângulos/tetraedros não-conforme.