\label{ch:conclusao}

Este trabalho apresentou uma metodologia para simulação de fumaça em malhas não-estruturadas bidimensionais utilizando o método RBF-FD. A abordagem proposta combina técnicas estabelecidas de simulação de fluidos baseadas no método \emph{Stable Fluids} com uma formulação que permite operar em malhas triangulares arbitrárias através de funções de base radial. Desenvolvemos soluções específicas para desafios inerentes a malhas não-estruturadas, incluindo o tratamento de fronteiras complexas através de nós fantasma temporários e um algoritmo eficiente para determinação de interseção raio-polígono durante a advecção semi-lagrangiana. Os resultados demonstraram que o método é capaz de reproduzir fenômenos físicos qualitativamente corretos, como a formação de estruturas vorticais e a interação adequada com fronteiras irregulares.

As principais contribuições deste trabalho incluem: (i) a adaptação do método \emph{Stable Fluids} para operar em malhas triangulares não-estruturadas utilizando RBF-FD; (ii) o desenvolvimento de uma técnica de nós fantasma temporários que preserva a estrutura original da malha e simplifica a implementação das condições de contorno; (iii) a implementação de um algoritmo eficiente de interseção raio-polígono otimizado com compilação JIT para viabilizar o \emph{backtracking} semi-lagrangiano em malhas arbitrárias; e (iv) a demonstração da viabilidade do método em domínios com geometrias irregulares, incluindo contornos geográficos reais.

\section{Publicações}
\label{sec:publicacoes}

Parte dos resultados preliminares desta pesquisa foi apresentada no XLI Congresso Nacional de Matemática Aplicada e Computacional (CNMAC) em 2023, realizado em Bonito, Mato Grosso do Sul. O trabalho intitulado ``Animação de fumaça em malhas não-estruturadas usando o método RBF-FD'' foi publicado nos anais do evento~\cite{Silva2023CNMAC} e apresentou os fundamentos da metodologia proposta, incluindo os primeiros experimentos com malhas triangulares não-estruturadas e a adaptação do método semi-lagrangiano para este contexto. A apresentação no CNMAC proporcionou feedback valioso da comunidade científica, contribuindo para o aprimoramento e refinamento da metodologia desenvolvida nesta dissertação.

\section{Desafios e Trabalhos Futuros}
\label{sec:trabalhos_futuros}

Durante o desenvolvimento deste trabalho, identificamos alguns desafios que representam oportunidades interessantes para investigações futuras. A conservação de massa ao longo de simulações prolongadas permanece como um aspecto que pode ser aprimorado, principalmente em regiões próximas às fronteiras onde a aproximação das condições de contorno introduz imprecisões. Este comportamento está relacionado à sensibilidade do método RBF-FD à proximidade entre nós e à escolha da função de base radial, particularmente a função poliarmônica utilizada neste trabalho.

Outra direção promissora para trabalhos futuros é a extensão do método para domínios não-simplesmente conexos, como geometrias com buracos ou múltiplas componentes conexas. Esta extensão requer o desenvolvimento de técnicas específicas para o tratamento das descontinuidades topológicas e das condições de contorno em fronteiras internas. Uma abordagem possível seria a decomposição do domínio em regiões simplesmente conexas ou a investigação de formulações alternativas para o cálculo dos pesos RBF-FD em topologias mais complexas.

O desempenho computacional do método também apresenta oportunidades de otimização. Embora tenhamos utilizado compilação JIT para acelerar operações críticas, a exploração de técnicas de paralelização mais sofisticadas poderia viabilizar simulações interativas, expandindo significativamente o potencial de aplicação do método. Além disso, a investigação de funções de base radial alternativas que apresentem melhor comportamento numérico em configurações geométricas desafiadoras pode aumentar a robustez do método.

Uma extensão natural deste trabalho seria a adaptação da metodologia para simulações tridimensionais. A transição de malhas triangulares para malhas tetraedrais mantém a estrutura conceitual do método RBF-FD, porém introduz desafios computacionais adicionais devido ao aumento significativo no número de graus de liberdade. Neste contexto, técnicas de refinamento adaptativo e paralelização se tornam ainda mais críticas para viabilizar simulações em tempo razoável. A extensão para 3D também permitiria explorar aplicações mais realistas em computação gráfica, como simulações de fumaça volumétrica em ambientes complexos.

Por fim, o desenvolvimento de técnicas de pré-processamento para otimizar a distribuição dos nós na malha e a implementação de métodos de correção de massa local representam caminhos naturais para o aprimoramento da precisão do método. Estas melhorias tornariam a abordagem ainda mais robusta e aplicável a uma gama mais ampla de problemas práticos em animação computacional, consolidando o método RBF-FD como uma alternativa viável para simulação de fluidos em domínios com geometrias complexas.
