\chapter{Dependências do Projeto}
\label{ap:dependencias}

Este apêndice lista as principais bibliotecas Python utilizadas na implementação do método proposto, juntamente com suas respectivas versões. A reprodução dos experimentos descritos neste trabalho requer a instalação destas dependências nas versões especificadas ou compatíveis.

\section{Bibliotecas Principais}

\begin{description}
    \item[NumPy 1.22.4] Biblioteca fundamental para computação numérica, utilizada para operações com arrays e álgebra linear.

    \item[SciPy 1.10.1] Biblioteca para computação científica, utilizada para resolver sistemas lineares esparsos (BiCGSTAB com precondicionador ILU), construção de matrizes esparsas e busca de vizinhos mais próximos (KD-Tree).

    \item[Numba 0.60] Compilador JIT (Just-In-Time) para Python, utilizado para otimização de operações numéricas intensivas, especialmente no algoritmo de intersecção raio-polígono.

    \item[Triangle 20230923] Wrapper Python para a biblioteca Triangle de triangulação de Delaunay, utilizada para geração de malhas triangulares a partir de contornos.

    \item[Matplotlib 3.7.1] Biblioteca de visualização, utilizada para geração de gráficos e análise dos resultados.

    \item[OpenCV-Python 4.7.0.72] Biblioteca de visão computacional, utilizada para processamento de imagens e gravação de vídeos da simulação.

    \item[PyOpenGL 3.1.7] Binding Python para OpenGL, utilizado para renderização em tempo real da simulação.

    \item[Shapely 2.0.1] Biblioteca para manipulação e análise de objetos geométricos planares.

    \item[libpysal 4.7.0] Biblioteca de análise espacial, utilizada para operações de vizinhança em malhas.

    \item[tqdm 4.65.0] Biblioteca para barras de progresso, utilizada durante a inicialização e processamento da simulação.
\end{description}

\section{Arquivo de Dependências Completo}

O arquivo \texttt{requirements.txt} completo, que pode ser utilizado para instalação automática via \texttt{pip install -r requirements.txt}, contém as seguintes especificações:

\begin{verbatim}
# Python 3.10.4
numpy<=1.22.4
scipy==1.10.1
numba==0.60
triangle==20230923
matplotlib==3.7.1
opencv-python==4.7.0.72
PyOpenGL==3.1.7
shapely==2.0.1
libpysal<=4.7.0
tqdm==4.65.0
Pillow==10.0.1
networkx==3.1
scikit-learn==1.2.2
Rtree==1.0.1
trimesh==3.22.0
Cython==0.29.35
\end{verbatim}

\section{Ambiente de Execução}

Os experimentos foram executados em um sistema Linux com Python 3.10.4. A utilização de versões diferentes das bibliotecas listadas pode resultar em comportamentos distintos ou incompatibilidades. Recomenda-se a criação de um ambiente virtual Python para isolar as dependências do projeto.
